\element{Program}

Description of the element {\tt Program} element:

\HIRContentsModel{ (typeTable, globalSymbols, globalDeclarations) }

\begin{HIRChildElements}
\HIRElementDef{typeTable}
{information on data type used by the program}{R}
\HIRElementDef{globalSymbols}
{information on global variables used by the program}{R}
\HIRElementDef{globalDeclarations}
{information about function and variable declarations}{R}
\end{HIRChildElements}

\begin{HIRAttributes}
\HIRAttrDef{compiler-info}{text}
{F-to-F translator information}{O}
\HIRAttrDef{version}{text}
{F-to-F translator version information}{O}
\HIRAttrDef{time}{text}
{Date and time of translation}{O}
\HIRAttrDef{language}{text}
{source language information}{O}
\HIRAttrDef{source}{text}
{source code information}{O}
\end{HIRAttributes}

\element{Dimension}

The {\tt Dimension} elements defines a dimension of a multidimensional space where fields are discretized and over which computations iterate. 

\HIRContentsModel{ () }
\begin{HIRAttributes}
	\HIRAttrDef{name}{text}
	{Name of the dimension}{R}
\end{HIRAttributes}

\section{ {\tt Type} element}

The {\tt Type} defines the type of storage declarations
\HIRContentsModel{ () }

\begin{HIRAttributes}
	\HIRAttrDef{name}{text}
	{any of the supported types (double|int|float)}{R}
\end{HIRAttributes}

\element{FieldDecl}
The {\tt FieldDecl} element defines a multidimensional field storage
\HIRContentsModel{ (Dimension+) }

\begin{HIRChildElements}
	\HIRElementDef{Dimension}
	{dimensions of the multidimensional space where the storage is defined}{R}
\end{HIRChildElements}

\begin{HIRAttributes}
	\HIRAttrDef{name}{text}
	{name of the field}{R}
\end{HIRAttributes}

\element{VarDecl}

The {\tt VarDecl} represents a N-dimensional scalar.
\HIRContentsModel{ (Type+) }

\begin{HIRChildElements}
	\HIRElementDef{Type}
	{type of the variable n-dimensional variable}{R}
\end{HIRChildElements}

\begin{HIRAttributes}
	\HIRAttrDef{name}{text}
	{name of the variable}{R}
	\HIRAttrDef{ndims}{int}
	{number of dimensions of the variable}{R}
	\HIRAttrDef{initialization}{string}
	{operation to initialize the variable}{0}
\end{HIRAttributes}

\element{ControlFlowGraph}

The {\tt ControlFlowGraph} contains the sequence of statements of a control flow of the program. 
\HIRContentsModel{ (VarDecl|IfStmt|BlockStmt|UnaryOp|BinaryOp|TernaryOp|AssignmentOp|Literal) }

\element{Literal}

The {\tt Literal} defines the specification of a literal. 
\HIRContentsModel{ (Type) }

\begin{HIRChildElements}
	\HIRElementDef{Type}
	{type of the literal}{R}
\end{HIRChildElements}

\begin{HIRAttributes}
	\HIRAttrDef{value}{int|double}
	{value of the literal}{R}
\end{HIRAttributes}

\element{AssignmentOp}

The {\tt AssignmentOp} defines an assignment operation expression. 

\subsubsection*{ContentsModel}{}

\begin{lstlisting}[style=default]
(Expr[lhs], Expr[rhs])

where if (scope == domain_computation)

[lhs] = (^\concept{VarDecl}^|^\concept{VarAccess}^|^\concept{FieldAccess}^),
[rhs] =	(^\concept{UnaryOp}^|^\concept{BinaryOp}^|^\concept{TernaryOp}^|^\concept{Literal}^|^\concept{FieldAccess}^)

else

[lhs] = (^\concept{VarDecl}^|^\concept{VarAccess}^)
[rhs] =	(^\concept{UnaryOp}^|^\concept{BinaryOp}^|^\concept{TernaryOp}^|^\concept{Literal}^)
\end{lstlisting}


\begin{HIRChildElements}
	\HIRElementDef{[lhs]}
	{left hand side expression of the binary operator}{R}
	\HIRElementDef{[rhs]}
	{right hand side expression of the binary operator}{R}
\end{HIRChildElements}


\element{BinaryOp}

The {\tt BinaryOp} defines an binary operator expression.
 
\subsubsection*{ContentsModel}{}

\begin{lstlisting}[style=default]
(Expr[lhs], Expr[rhs])

where if (scope == domain_computation)
 
[lhs] = (^\concept{UnaryOp}^|^\concept{BinaryOp}^|^\concept{TernaryOp}^|^\concept{FieldAccess}^|^\concept{VarAccess}^|^\concept{Literal}^)
[rhs] =	(^\concept{UnaryOp}^|^\concept{BinaryOp}^|^\concept{TernaryOp}^|^\concept{FieldAccess}^|^\concept{VarAccess}^|^\concept{Literal}^)

else

[lhs] = (^\concept{UnaryOp}^|^\concept{BinaryOp}^|^\concept{TernaryOp}^|^\concept{VarAccess}^|^\concept{Literal}^)
[rhs] = (^\concept{UnaryOp}^|^\concept{BinaryOp}^|^\concept{TernaryOp}^|^\concept{VarAccess}^|^\concept{Literal}^)
\end{lstlisting}


\begin{HIRChildElements}
	\HIRElementDef{[lhs]}
	{left hand side expression of the binary operator}{R}
	\HIRElementDef{[rhs]}
	{right hand side expression of the binary operator}{R}
\end{HIRChildElements}

\begin{HIRAttributes}
	\HIRAttrDef{operator}{string}
	{operator being applied to the operands}{R}
\end{HIRAttributes}

\element{UnaryOp}

The {\tt UnaryOp} defines an unary operator expression.

\subsubsection*{ContentsModel}{}

\begin{lstlisting}[style=default]
(Expr)

where if (scope == domain_computation)

[Expr] = (^\concept{VarDecl}^|^\concept{VarAccess}^|^\concept{FieldAccess}^),

else

[Expr] = (^\concept{VarDecl}^|^\concept{VarAccess}^|)
\end{lstlisting}


\begin{HIRChildElements}
	\HIRElementDef{[lhs]}
	{left hand side expression of the binary operator}{R}
	\HIRElementDef{[rhs]}
	{right hand side expression of the binary operator}{R}
\end{HIRChildElements}

\begin{HIRAttributes}
	\HIRAttrDef{operator}{string}
	{operator being applied to the operands}{R}
\end{HIRAttributes}

\element{TernaryOp}

The {\tt TernaryOp} defines an ternary operator expression.

\subsubsection*{ContentsModel}{}

\begin{lstlisting}[style=default]
(Expr[cond], Expr[lhs], Expr[rhs])

where if (scope == domain_computation)
[cond] = (^\concept{UnaryOp}^|^\concept{BinaryOp}^|^\concept{TernaryOp}^|^\concept{FieldAccess}^|^\concept{VarAccess}^|^\concept{Literal}^)
[lhs] = (^\concept{UnaryOp}^|^\concept{BinaryOp}^|^\concept{TernaryOp}^|^\concept{FieldAccess}^|^\concept{VarAccess}^|^\concept{Literal}^),
[rhs] =	(^\concept{UnaryOp}^|^\concept{BinaryOp}^|^\concept{TernaryOp}^|^\concept{FieldAccess}^|^\concept{VarAccess}^|^\concept{Literal}^)

else

[cond] = (^\concept{UnaryOp}^|^\concept{BinaryOp}^|^\concept{TernaryOp}^|^\concept{VarAccess}^|^\concept{Literal}^)
[lhs] = (^\concept{UnaryOp}^|^\concept{BinaryOp}^|^\concept{TernaryOp}^|^\concept{VarAccess}^|^\concept{Literal}^)
[rhs] = (^\concept{UnaryOp}^|^\concept{BinaryOp}^|^\concept{TernaryOp}^|^\concept{VarAccess}^|^\concept{Literal}^)
\end{lstlisting}

\begin{HIRChildElements}
	\HIRElementDef{[cond]}
	{expression that defines the condition of the ternary operator}{R}
	\HIRElementDef{[lhs]}
	{left hand side expression of the binary operator}{R}
	\HIRElementDef{[rhs]}
	{right hand side expression of the binary operator}{R}
\end{HIRChildElements}

\begin{HIRAttributes}
	\HIRAttrDef{operator}{string}
	{operator being applied to the operands}{R}
\end{HIRAttributes}


\element{VarAccess}
The {\tt VarAccess} is a expression that defines an access to a \concept{VarDecl}

\subsubsection*{ContentsModel}{}

\begin{lstlisting}[style=default]
(^\concept{VarDecl}^,^\concept{Literal}^+)
\end{lstlisting}

\begin{HIRChildElements}
	\HIRElementDef{VarDecl}
	{The var declaration that is being accessed in this expression}{R}
	\HIRElementDef{Literal}
	{access index of the var, when it is declared with more than 1 dimension}{O}
\end{HIRChildElements}


\element{FieldAccess}
 The {\tt FieldAccess} is a expression that defines an access to a field
 
\subsubsection*{ContentsModel}{}

\begin{lstlisting}[style=default]
(^\concept{FieldDecl}^,^\concept{Offset}^+)
\end{lstlisting}

\begin{HIRChildElements}
	\HIRElementDef{FieldDecl}
	{The field decl that is being accessed in this expression}{R}
	\HIRElementDef{Offset}
	{An offset (relative to current grid position) used to de-reference the field access}{O}
\end{HIRChildElements}


\element{Offset}
The {\tt Offset} is the relative distance in a given \concept{Dimension} to a neighbor grid point.

\subsubsection*{ContentsModel}{}

\begin{lstlisting}[style=default]
(^\concept{Dimension}^)
\end{lstlisting}

\begin{HIRChildElements}
	\HIRElementDef{Dimension}
	{Identifies the dimension where the offset is computed}{R}
\end{HIRChildElements}

\begin{HIRAttributes}
	\HIRAttrDef{distance}{int}
	{relative distance in the given dimension of the offset}{R}
\end{HIRAttributes}

\element{BoundaryCondition}
The {\tt BoundaryCondition} defines the strategy to apply a boundary condition to a field, if required.

\subsubsection*{ContentsModel}{}

\begin{lstlisting}[style=default]
(^\concept{FieldDecl}^+ )
\end{lstlisting}

\begin{HIRChildElements}
	\HIRElementDef{Dimension}
	{Identifies the dimension where the offset is computed}{R}
\end{HIRChildElements}

\begin{HIRAttributes}
	\HIRAttrDef{distance}{int}
	{relative distance in the given dimension of the offset}{R}
\end{HIRAttributes}
