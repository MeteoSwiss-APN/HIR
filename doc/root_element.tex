\section{General HIR elements}

\element{Program}

Description of the element {\tt Program} element:

\subsubsection*{ContentsModel}{}

\begin{lstlisting}[style=default]
(^\concept{GridDimension}^[dimension]+,^\concept{Domain}^[domain], ^\concept{FieldDecl}^[field]+, ^\concept{VarDecl}^[vararg]*, (^\concept{ScopedProgram}^|^\concept{ExternalKernel}^)[scope]+)
\end{lstlisting}

\begin{HIRChildElements}
\HIRElementDef{dimension}
{Definition of all the dimensions used within the program}{R}
\HIRElementDef{domain}
{specifies a domain that serves as hints to the compiler}{R}
\HIRElementDef{field}
{Definition of all the fields used by the program}{R}
\HIRElementDef{vararg}
{Definition of scalar variables arguments to the program}{O}
\HIRElementDef{ScopedProgram[scope]}
{a scoped program with computational patterns supported by the concepts of the HIR}{O}
\HIRElementDef{ExternalKernek[scope]}
{describes a kernel that contains computational patterns non supported by the concepts of the HIR}{O}
\end{HIRChildElements}

\begin{HIRAttributes}
\HIRAttrDef{HIRversion}{text}
{version of the HIR}{R}
\HIRAttrDef{DomainPolicy}{text}
{policy that defines the domain of the HIR}{R}
\HIRAttrDef{time}{text}
{Date and time of translation}{O}
\HIRAttrDef{language}{text}
{source language information}{O}
\HIRAttrDef{source}{text}
{source code information}{O}
\end{HIRAttributes}

\element{Domain}

The domain provides domain information of the application that is used as hints to the compiler toolchain.

\subsubsection*{ContentsModel}{}

\begin{lstlisting}[style=default]
((^\concept{GridDimension}^[domain_parallel_dim])+, ^\concept{GridDimension}^[vertical_dim], (^\concept{GridDimension}^[parallel_dim])+
\end{lstlisting}

\begin{HIRChildElements}
	\HIRElementDef{domain\_parallel\_dim}
	{specifies the dimensions over which the domain is parallelized}{R}
	\HIRElementDef{vertical\_dim}
	{specifies the vertical dimension}{R}
	\HIRElementDef{parallel\_dim}
	{specifies dimensions on which computations are embarrassingly parallel}{R}
\end{HIRChildElements}

\element{ScopedProgram}

The {\tt ScopedProgram} defines all the computations performed using concepts of the HIR

\subsubsection*{ContentsModel}{}

\begin{lstlisting}[style=default]
(^\concept{BlockStmt}^)
\end{lstlisting}

\begin{HIRChildElements}
	\HIRElementDef{BlockStmt}
	{Block statement containing the sequence of statements that forms the computation of the program}{R}
\end{HIRChildElements}

\element{ExternalKernel}

The {\tt ExternalKernel} defines a call to an external kernel, for which computations description is not provided

\subsubsection*{ContentsModel}{}

\begin{lstlisting}[style=default]
(^\concept{FieldDecl}^[input]+, ^\concept{FieldDecl}^[output]+)
\end{lstlisting}

\begin{HIRChildElements}
	\HIRElementDef{input}
	{defines the list of input fields}{R}
	\HIRElementDef{output}
	{defines the list of output fields}{R}

\end{HIRChildElements}

\element{GridDimension}

The {\tt GridDimension} elements defines a dimension of a multidimensional space where fields are discretized and over which \concept{Computation}s iterate.

\HIRContentsModel{ () }
\begin{HIRAttributes}
	\HIRAttrDef{name}{text}
	{Name of the dimension}{R}
\end{HIRAttributes}

\element{DimensionLevel}

The {\tt DimensionLevel} it is used to specify a position (that is specified as a runtime argument to the \concept{Program}) in a given dimension

\subsubsection*{ContentsModel}{}

\begin{lstlisting}[style=default]
(^\concept{VarAccess}^[var_placeholder], ^\concept{Literal}^[offset])
\end{lstlisting}

\begin{HIRChildElements}
	\HIRElementDef{var\_placeholder}
	{It uses a scalar variable of rank N where each element act as a marker, whose runtime values will store the positions within the extent of the dimension.}{R}
	\HIRElementDef{offset}
	{It is a integer offset that shifts the position of the level with respect to the value of the var\_placeholder }{R}
\end{HIRChildElements}

\element{DimensionInterval}

The {\tt DimensionInterval} defines an interval on a dimension.

\subsubsection*{ContentsModel}{}

\begin{lstlisting}[style=default]
(^\concept{DimensionLevel}^,^\concept{DimensionLevel}^)
\end{lstlisting}

\begin{HIRChildElements}
	\HIRElementDef{DimensionLevel}
	{position placeholder on a dimension that defines the begin and end of the interval}{R}
\end{HIRChildElements}

\element{Type}

The {\tt Type} defines the type of storage declarations
\HIRContentsModel{ () }

\begin{HIRAttributesVal}
	\HIRAttrValDef{name}{text}
	{any of the supported types}{(double,int,float)}{R}
\end{HIRAttributesVal}

\element{FieldDecl}
The {\tt FieldDecl} element defines a multidimensional field storage
\subsubsection*{ContentsModel}{}

\begin{lstlisting}[style=default]
(^\concept{Type}^,^\concept{GridDimension}^+)
\end{lstlisting}

\begin{HIRChildElements}
	\HIRElementDef{GridDimension}
	{dimensions of the multidimensional space where the storage is defined}{R}
	\HIRElementDef{Type}
	{value type of the grid elements of the field}{R}
\end{HIRChildElements}

\begin{HIRAttributes}
	\HIRAttrDef{name}{text}
	{name of the field}{R}
\end{HIRAttributes}


\element{Offset}
The {\tt Offset} is the relative distance in a given \concept{GridDimension} to a neighbor grid point.

\subsubsection*{ContentsModel}{}

\begin{lstlisting}[style=default]
(^\concept{GridDimension}^)
\end{lstlisting}

\begin{HIRChildElements}
	\HIRElementDef{GridDimension}
	{Identifies the dimension where the offset is computed}{R}
\end{HIRChildElements}

\begin{HIRAttributes}
	\HIRAttrDef{distance}{int}
	{relative distance in the given dimension of the offset}{R}
\end{HIRAttributes}

\element{Computation}
The {\tt Computation} defines an iteration loop over the specified \concept{GridDimension}s of the domain.

\subsubsection*{ContentsModel}{}

\begin{lstlisting}[style=default]
((^\concept{GridDimension}^|^\concept{DimensionInterval}^)[dimension]+,(^\concept{BlockStmt}^))
\end{lstlisting}

\begin{HIRChildElements}
	\HIRElementDef{GridDimension[dimension]}
	{Specifies the dimensions where the computation is defined,
		convering the whole extent of the grid for that dimension}{O}
	\HIRElementDef{DimensionInterval[dimension]}
	{Provides a specific range on a dimension to iterate over}{O}
	\HIRElementDef{BlockStmt}
	{Specifies the block with the list of statements that form the computation}{R}
\end{HIRChildElements}


\element{BoundaryCondition}
The {\tt BoundaryCondition} defines the strategy to apply a boundary condition to a field, if required.

\subsubsection*{ContentsModel}{}

\begin{lstlisting}[style=default]
(^\concept{FieldDecl}^, ^\concept{BlockStmt}^)
\end{lstlisting}

\begin{HIRChildElements}
	\HIRElementDef{FieldDecl}
	{field subject of boundary condition}{R}
	\HIRElementDef{BlockStmt}
	{BlockStmt with statement that implement the boundary condition computation}{R}

\end{HIRChildElements}

\begin{HIRAttributes}
	\HIRAttrDef{distance}{int}
	{relative distance in the given dimension of the offset}{R}
\end{HIRAttributes}
