\section{Introduction}


This document provides a full specification of the high-level intermediate representation (HIR) for weather and climate applications.

\subsection{Conventions}

Some of the elements will allow to have different children type of nodes depending on the scope of the node.

There are two scopes: 
\begin{enumerate}
	\item {\tt control\_flow} defines the scope of nodes where one or more of the parallel dimensions of the \concept{Domain} are not yet resolved within a \concept{Computation}
	\item {\tt domain\_computation} is the scope of nodes where all the parallel dimensions of the \concept{Domain} are resolved within a \concept{Computation}
\end{enumerate}

The {\bf ContentsModel} section of each node describes, using regular expression, that supported children nodes. 
Some of the children are named nodes in order to identify in the specification of the node. The following example shows
a node with two children, identified by {\tt lhs} and {\tt rhs}, where the {\tt rhs} can be either a \concept{FieldAccess} or a  \concept{Literal}

\subsubsection*{ContentsModel}{}

\begin{lstlisting}[style=default,frame=none]
(^\concept{FieldAccess}^, (^\concept{VarAccess}^|^\concept{Literal}^))
\end{lstlisting}

The children elements are described as R (required)/O (optional)/A (any of several options).
The previous example would define the children as 

\begin{HIRChildElements}
	\HIRElementDef{FieldAccess}
	{description of semantic of child}{R}
	\HIRElementDef{VarAccess}
	{description of semantic of child}{A}
		\HIRElementDef{Literal}
	{description of semantic of child}{A}
\end{HIRChildElements}
