\section{Introduction}

\subsection{Background}
The deliverable D2.1 presented a comprehensive 
set of computational patterns derived from dwarfs and representative 
computations of the models participating in the ESCAPE-2 project.
Each computational pattern derived a set of language elements
that would be necessary to express that type of computations
in a way that language elements are minimal, orthogonal, simple 
and non redundant. 

This document provides a full and formal specification of 
the high-level intermediate representation (HIR) for weather and climate applications.
A full specification is presented as a set of concepts following
the language elements identified in D2.1. They
are organized as a tree to represent a full program. 
Each concept captures a specific information of a the computational
patterns supported by the DSL toolchain.
The semantic of each node is described as well as its properties 
and children nodes that it supports. 
The HIR serves as an interface between a language frontend (D2.2)
and the implementation of a compiler toolchain that
generates an efficient, parallel and optimized implementation 
of a model described using a concise and compact frontend language. 


\section{Scope of this deliverable}
\subsection{Objectives of this deliverable} 
The objective of this deliverable is to provide a technical 
and formal specification of a full high-level intermediate
representation.
\subsection{Work performed in this deliverable} 
A comprehensive set of concepts have been discussed and formalized
in this document as a proposal for a standard intermediate representation. 
A large fraction of the specification has been prototyped within
the Dawn compiler, therefore proving that they are implementable and
can feed an efficient DSL compiler toolchain. 
Additionally a frontend has been developed (D2.3) with a concise
language for numerical developments of weather codes, that is
capable of translating the frontend language into an intermediate 
representation. 
\subsection{Deviations and counter measures} 
There were no deviations from the project proposal for this deliverable.
\subsection{Conventions}

Some of the elements will allow to have different children type of nodes depending on the scope of the node.

We defined two scopes: 
\begin{enumerate}
	\item {\tt control\_flow} defines the scope of nodes where one or more of the parallel dimensions of the \concept{Domain} are not yet resolved within a \concept{Computation}
	\item {\tt domain\_computation} is the scope of nodes where all the parallel dimensions of the \concept{Domain} are resolved within a \concept{Computation}
\end{enumerate}

The scope is defined in order to constrain the validity of certain nodes.
For example, a \concept{FieldAccess} will not be valid in the 
{\tt control\_flow} since we need to resolve all the grid dimensions
in order to be able to access a field for a grid point.


The {\bf ContentsModel} section of each node describes, using regular expressions, the type of nodes that are expected/supported for its children nodes. 
Children nodes are identified by position. 

The following example shows
a node with two children ({\tt lhs} and {\tt rhs}) where 
the {\tt lhs} is a \concept{FieldAccess} and the {\tt rhs} can be either a \concept{VarAccess} or a  \concept{Literal}

\subsubsection*{ContentsModel}{}

\begin{lstlisting}[style=default,frame=none]
(^\concept{FieldAccess}^, (^\concept{VarAccess}^|^\concept{Literal}^))
\end{lstlisting}

The semantic and properties of the children elements are 
described in a following table, where the property is defined 
as R (required)/O (optional)/A (any of several options).
The previous example would describe its children as 

\begin{HIRChildElements}
	\HIRElementDef{FieldAccess}
	{description of semantic of child}{R}
	\HIRElementDef{VarAccess}
	{description of semantic of child}{A}
		\HIRElementDef{Literal}
	{description of semantic of child}{A}
\end{HIRChildElements}

This version of the specification supports two category of models, 
according to the horizontal grid employed: Cartesian grids and
irregular grids (like icosahedral or octahedral grids). 
Most of the elements are common, however some are only 
supported for the irregular grids category. 

Unless otherwise specified, an element will be considered 
shared for the two categories.
References to elements of the two distinct categories 
are highlighted using different colors. 
For example the following content model: 

\subsubsection*{ContentsModel}{}

\begin{lstlisting}[style=default,frame=none]
(^\concept{FieldAccess}^, (^\irrconcept{LocationType}^))
\end{lstlisting}

requires a \concept{FieldAccess} node and additionall, 
only for irregular grid models, a \irrconcept{LocationType}
node.