\section{Expression elements}

\element{Literal}

The {\tt Literal} defines the specification of a literal.
\HIRContentsModel{ (Type) }

\begin{HIRChildElements}
	\HIRElementDef{Type}
	{type of the literal}{R}
\end{HIRChildElements}

\begin{HIRAttributes}
	\HIRAttrDef{value}{string}
	{value of the literal}{R}
\end{HIRAttributes}

\element{Binary operations}

The elements representing binary operators are described below and follows the
\concept{binaryOpModel}: 

\begin{tabular}{|l|p{4cm}|l|}
	\hline
	 element & operator & operation \\ \hline\hline
	 {\tt plusOp} & + & addition \\ \hline 
	 {\tt minusOp} & - & subtraction \\ \hline 
	 {\tt mulOp} & * & multiplication \\ \hline 
	 {\tt divOp} & / & division \\ \hline 
	 {\tt powerOp} & ** & power \\ \hline 
	 {\tt logicalAnd} & \&\& & logical AND \\ \hline 
	 {\tt logicalOr} & || & logical OR \\ \hline 
	 {\tt logicalEqual} & == & logical equality \\ \hline 
	 {\tt logicalNotEqual} & != & logical unequality \\ \hline 
	 {\tt logicalGt} & > & logical greater than \\ \hline 
	 {\tt logicalLt} & < & logical less than \\ \hline 
	 {\tt logicalGe} & >= & logical greater or equal than \\ \hline 
	 {\tt logicalLe} & <= & logical less or equal than \\ \hline 
\end{tabular}


\element{Unary operations}

The elements representing unary operators are described below and follows the
\concept{unaryOpModel}:

\begin{tabular}{|l|p{4cm}|l|}
	\hline
	 element & operator & operation \\ \hline\hline
	 {\tt unaryMinus} & - & sign inversion \\ \hline 
	 {\tt logNot} & ! & logical not \\ \hline 
	 {\tt incrementOp} & ++ & increment by one \\ \hline 
	 {\tt decrementOp} & -- & decrement by one \\ \hline 
\end{tabular}

\subsubsection*{ContentsModel}{}

\begin{lstlisting}[style=default,frame=none]
(exprModel)
\end{lstlisting}

\begin{HIRChildElements}
	\HIRElementDef{exprModel}
	{Specifies the operand expression. Refer to \concept{exprModel}}{R}
\end{HIRChildElements}

\element{TernaryOp}

The {\tt TernaryOp} defines an ternary operator expression.

\subsubsection*{ContentsModel}{}

\begin{lstlisting}[style=default,frame=none]
(exprModel, exprModel, exprModel)
\end{lstlisting}

\begin{HIRChildElements}
	\HIRElementDef{exprModel}
	{Specifies the condition of the operation as the first operand, the 
	 left-hand expression of the ternary operation as the second 
	 operand, the right-hand expression of the ternary operation as the third 
	 operand. Refer to \concept{exprModel}.}{R}
\end{HIRChildElements}

\begin{HIRAttributes}
	\HIRAttrDef{operator}{string}
	{operator being applied to the operands}{R}
\end{HIRAttributes}

\element{FctCall}
Built-in function call for mathematical functions.


\subsubsection*{ContentsModel}{}
\begin{lstlisting}[style=default,frame=none]
(exprModel+)
\end{lstlisting}


\begin{HIRChildElements}
	\HIRElementDef{exprModel}{Arguments of the function call.}{R}
\end{HIRChildElements}

\begin{HIRAttributes}
	\HIRAttrDef{name}{string}
	{function name: (abs, sqrt, sin, cos, tan, asin, acos, atan, exp, log)}{R}
\end{HIRAttributes}

\element{VarAccess}
The {\tt VarAccess} is a expression that defines an access to a \concept{VarDecl}

\subsubsection*{ContentsModel}{}

\begin{lstlisting}[style=default,frame=none]
(^\concept{Literal}^*)
\end{lstlisting}

\begin{HIRChildElements}
	\HIRElementDef{Literal}
	{access index of the var, when it is declared with more than 1 dimension}{O}
\end{HIRChildElements}

\begin{HIRAttributes}
	\HIRAttrDef{name}{string}
	{The var declaration that is being accessed in this expression}{R}
\end{HIRAttributes}

\element{FieldAccess}
The {\tt FieldAccess} is a expression that defines an access to a field

\subsubsection*{ContentsModel}{}

\begin{lstlisting}[style=default,frame=none]
(^\concept{Offset}^+)
\end{lstlisting}

\begin{HIRChildElements}
	\HIRElementDef{Offset}
	{An offset (relative to current grid position) used to de-reference the field access}{O}
\end{HIRChildElements}
\begin{HIRAttributes}
	\HIRAttrDef{name}{string}
	{The name of the field declaration that is being accessed in this expression}{R}
\end{HIRAttributes}


