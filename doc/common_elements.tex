\section{Common elements}
The definitions in this section are aliases to define nodes 
commonly used in the documentation. 
They do not exist as real nodes of the specification. 
Any entry of a Model node found in the documentation must be replace by the 
equivalent list of nodes provided here. 

\model{stmtModel}
The {\tt stmtModel} is commonly used for elements that refer statements.

\subsubsection*{ContentsModel}{}

\begin{lstlisting}[style=default,frame=none]
if (scope == domain_computation)    
  (^\concept{VarDecl}^|^\concept{IfStmt}^|^\concept{SparseIfStmt}^|^\concept{BlockStmt}^|^\concept{LoopOn}^)
else
  (^\concept{VarDecl}^|^\concept{IfStmt}^|^\concept{BlockStmt}^|^\concept{VerticalRegion}^)
\end{lstlisting}


\model{exprModel}
The {\tt exprModel} is commonly used for elements that refer expression.

\subsubsection*{ContentsModel}{}

\begin{lstlisting}[style=default,frame=none]
if (scope == domain_computation)    
    (^\concept{unaryOpModel}^|^\concept{binaryOpModel}^|^\concept{TernaryOp}^|^\concept{Literal}^|^\concept{FieldAccess}^|^\concept{VarAccess}^|^\concept{FctCall}^|^\irrconcept{NeighbourReduce}^)
else
    (^\concept{unaryOpModel}^|^\concept{binaryOpModel}^|^\concept{TernaryOp}^|^\concept{VarAccess}^|^\concept{Literal}^|^\concept{FctCall}^)
\end{lstlisting}


\model{lValueModel}
The {\tt lValueModel} is commonly used for elements that refer left-hand side
expression.

\subsubsection*{ContentsModel}{}

\begin{lstlisting}[style=default,frame=none]
where if (scope == domain_computation)    
    (^\concept{VarDecl}^|^\concept{VarAccess}^|^\concept{FieldAccess}^)
else
    (^\concept{VarDecl}^|^\concept{VarAccess}^)
\end{lstlisting}


\model{binaryOpModel}
The {\tt binaryOpModel} is used for elements that refer to binary operations.


\subsubsection*{ContentsModel}{}

\begin{lstlisting}[style=default,frame=none]
(exprModel, exprModel)
\end{lstlisting}

\begin{HIRChildElements}
	\HIRElementDef{exprModel}
	{Specifies the left-hand expression as the first operand, the right-hand 
	 expression as the second operand. Refer to \concept{exprModel}.}{R}
\end{HIRChildElements}

\model{unaryOpModel}
The {\tt unaryOpModel} is used for elements that refer to unary operations.


\subsubsection*{ContentsModel}{}

\begin{lstlisting}[style=default,frame=none]
(exprModel)
\end{lstlisting}

\begin{HIRChildElements}
	\HIRElementDef{exprModel}
	{Specifies the right-hand expression. Refer to \concept{exprModel}.}{R}
\end{HIRChildElements}


\model{Common attributes}
Some elements may have the following attributes. 
\begin{HIRAttributes}
    \HIRAttrDef{lineno}{text}
    {Specifies the line number in the source program}{R}
    \HIRAttrDef{file}{text}
    {Specifies the source code file name}{R}
\end{HIRAttributes}

