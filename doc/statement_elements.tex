\section{Statement elements}

\element{LoopOn}

The {\tt LoopOn} describes an explicit loop on a sparse dimension of
fields (opposite to the \concept{Computation} node that is 
used only for dense dimensions).
The separation is made to distinguish between loops around 
dense dimensions where the parallelization is performed and sparse 
dimensions not affected by that.  

\subsubsection*{ContentsModel}{}

\begin{lstlisting}[style=default,frame=none]
((^\concept{GridDimension}^)+,(^\concept{BlockStmt}^))
\end{lstlisting}

\begin{HIRChildElements}
	\HIRElementDef{GridDimension}
	{Specifies the sparse dimensions where the computation is defined,
		convering the whole extent of the grid for that dimension}{O}
	\HIRElementDef{BlockStmt}
	{Specifies the block with the list of statements that form the computation}{R}
\end{HIRChildElements}


\element{VarDecl}

The {\tt VarDecl} represents a N-dimensional scalar.
\HIRContentsModel{ (Type) }

\begin{HIRChildElements}
	\HIRElementDef{Type}
	{type of the variable n-dimensional variable}{R}
\end{HIRChildElements}

\begin{HIRAttributes}
	\HIRAttrDef{name}{text}
	{name of the variable}{R}
	\HIRAttrDef{ndims}{int}
	{number of dimensions of the variable}{R}
	\HIRAttrDef{isarg}{bool}
    {specifies if the variable is argument to the main program}{R}
	\HIRAttrDef{initialization}{string}
	{operation to initialize the variable}{0}
\end{HIRAttributes}

\element{IfStmt}
The {\tt IfStmt} is a statement element that defines an if condition.

\subsubsection*{ContentsModel}{}

\begin{lstlisting}[style=default,frame=none]
( Condition, Then, Else? )

where if (scope == domain_computation)
Condition = (^\concept{unaryOpModel}^|^\concept{binaryOpModel}^|^\concept{TernaryOp}^|^\concept{FieldAccess}^|^\concept{VarAccess}^|^\concept{Literal}^|^\irrconcept{NeighbourReduce}^)
else
Condition = (^\concept{unaryOpModel}^|^\concept{binaryOpModel}^|^\concept{TernaryOp}^|^\concept{VarAccess}^|^\concept{Literal}^)
\end{lstlisting}

\begin{HIRChildElements}
	\HIRElementDef{Condition}
	{contains expression that defines the condition of the if block}{R}
	\HIRElementDef{Then}
	{contains a sequence of \concept{stmtModel} that compose the {\tt then} 
	 computation of the block}{R}
	\HIRElementDef{Else}
	{contains a sequence of \concept{stmtModel} that compose the {\tt Else} 
	 computation of the block}{O}
\end{HIRChildElements}

\element{SparseIfStmt}
The {\tt SparseIfStmt} represents the equivalent of an \concept{IfStmt} where the condition is seldom satisfied.

\subsubsection*{ContentsModel}{}

\begin{lstlisting}[style=default,frame=none]
( Condition, Then, Else? )

where if (scope == domain_computation)
Condition = (^\concept{unaryOpModel}^|^\concept{binaryOpModel}^|^\concept{TernaryOp}^|^\concept{FieldAccess}^|^\concept{VarAccess}^|^\concept{Literal}^|^\irrconcept{NeighbourReduce}^)
else
Condition = (^\concept{unaryOpModel}^|^\concept{binaryOpModel}^|^\concept{TernaryOp}^|^\concept{VarAccess}^|^\concept{Literal}^)
\end{lstlisting}

\begin{HIRChildElements}
	\HIRElementDef{Condition}
	{contains expression that defines the condition of the if block}{R}
	\HIRElementDef{Then}
	{contains a sequence of \concept{stmtModel} that compose the {\tt then} 
		computation of the block}{R}
	\HIRElementDef{Else}
	{contains a sequence of \concept{stmtModel} that compose the {\tt Else} 
		computation of the block}{O}
\end{HIRChildElements}

The {\tt SparseIfStmt} is only valid in a {\tt domain\_computation} scope.

\element{BlockStmt}
The {\tt BlockStmt} is a statement element that defines an block (of statements).

\subsubsection*{ContentsModel}{}

\begin{lstlisting}[style=default,frame=none]
(^\concept{stmtModel}^+)
\end{lstlisting}

\begin{HIRChildElements}
	\HIRElementDef{stmtModel}
	{statement that composes the {\tt block} computation}{R}
\end{HIRChildElements}

\element{AssignmentStmt}

The {\tt AssignmentStmt} defines an assignment operation expression.

\subsubsection*{ContentsModel}{}

\begin{lstlisting}[style=default,frame=none]
(^\concept{lValueModel}^, ^\concept{exprModel}^ )
\end{lstlisting}


\begin{HIRChildElements}
	\HIRElementDef{lValueModel}
	{Specifies the left-hand side expression. 
	 Refer to \concept{lValueModel}.}{R}
	\HIRElementDef{exprModel}
	{Specifies the right-hand side expression. 
	 Refer to \concept{exprModel}.}{R}
\end{HIRChildElements}

